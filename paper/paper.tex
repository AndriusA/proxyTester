% This version uses the latex2e styles, not the very ancient 2.09 stuff.
\documentclass{sig-alternate-10pt}
\usepackage{endnotes,url}
%\usepackage{sig-alternate-10pt,endnotes}
\usepackage{epsfig}
\usepackage{amssymb}
\usepackage{amsmath}
\usepackage{algorithm}
\usepackage{algorithmic}
\usepackage{amsfonts}
\usepackage{graphicx}
\usepackage{verbatim}
% Replaced by subfig package
% \usepackage{subfigure}
\usepackage{subfig}
\usepackage{epstopdf}
\usepackage{cite}
\usepackage{array}
\usepackage{multirow}
\usepackage{listings}


%\pdfpagewidth 8.5in
%\pdfpageheight 11in
%\baselineskip=12.5pt
%\setlength{\pdfpagewidth}{8.5in}
%\setlength{\pdfpageheight}{11in}
%\usepackage{pdfdraftcopy}

% To import classical math notation used

% Mathematical LaTeX formatting shortcuts
\newcommand{\midwor}[1]{\;\textnormal{ #1 }\;} % text in math mode
\newcommand{\ds}{\displaystyle} % shorcut to toggle display
	% All formula should punctuation as they are part of the text.
\newcommand{\vf}{\,,\,} % a comma with space
\newcommand{\ff}{\,.} % a dot 
	% The following command allows to write quickly sets { }   
\newcommand{\lset}{\left\{\left.\;}
\newcommand{\midset}{\;\right|\;\left.}
\newcommand{\dimset}{\right.\;\left|\;}
\newcommand{\rset}{\;\right.\right\}}
\newcommand{\jset}{\right.\right.\\ & \left.\left.}

% Number sets and usual operations
\newcommand{\Real}{\mathbb{R}} % real numbers
\newcommand{\RelInt}{\mathbb{Z}} % relative integer numbers
\newcommand{\NatInt}{\mathbb{N}} % natural integer numbers
\newcommand{\Compl}{\mathbb{C}} % complex numbers
\newcommand{\valent}[1]{\left\lfloor #1 \right\rfloor} % integer value
\newcommand{\valentsup}[1]{\left\lceil #1 \right\rceil} % integer value + 1
\newcommand{\eps}{\varepsilon} % very small real number
\newcommand{\argmin}{\textrm{argmin}} %argument of the minimum
\newcommand{\pospar}[1]{\left( #1 \right)^+} % positive part

% Probability notation
\newcommand{\proba}{\mathbb{P}} % Probability
\newcommand{\probaof}[1]{\mathbb{P}\left.\left[#1\right.\right]} % Probability of an event
\newcommand{\cond}{\right|\left.} % Condition on an event 
\newcommand{\expec}[1]{\mathbb{E}\left[#1\right]} % Expectation 
\newcommand{\vari}[1]{\textrm{V}\left[#1\right]} % Variance 
\newcommand{\Lun}{\mathbb{L}^{1}} % Space of integrable function
\newcommand{\as}{\;\textrm{a.s.}\;}
\newcommand{\ind}[1]{\mathbb{I}_{\{#1\}}} % Indicator
\newcommand{\tribF}{\mathcal{F}} % sigma-algebra

% Constants
\newcommand{\cst}{\textrm{\texttt{cst}}}

% Advanced probability 
\newcommand{\comp}{\circ} % composition of function
\newcommand{\shi}{\theta} % ergodic shift
\newcommand{\lstoc}{\displaystyle \leq_{\textrm{st}}} % Stochastic Order
\newcommand{\Ppalmof}[1]{\mathbb{P}^{0}_{#1}}
\newcommand{\Pofpalmof}[2]{\mathbb{P}^{0}_{#1}\left[#2\right]}
\newcommand{\expPalm}[1]{\mathbb{E}^{0}[#1]} % Palm Expectation
\newcommand{\expPalmof}[2]{\mathbb{E}^{0}_{#1}[#2]} % Palm Expectation w.r.t. point process f 

% (max,+) algebra
\newcommand{\Rmax}{\mathbb{R}_{\max}}
\newcommand{\ninf}{\varepsilon}
\newcommand{\matninf}{\textrm{\Large{$\ninf$}}}
\newcommand{\zer}{e}
\newcommand{\veczer}{\textrm{\Large{e}}}
\newcommand{\bigmax}{\bigvee}
\newcommand{\matId}{{\mathbb{I}}}
%\newcommand{\norm}[1]{\| #1 \|}
\newcommand{\linf}[1]{\left|\left| #1 \right|\right|}

% Matrices
\newcommand{\matA}{\mathbb{A}}
\newcommand{\matB}{\mathbb{B}}
\newcommand{\matC}{\mathbb{C}}
\newcommand{\matP}{\mathbb{P}}
\newcommand{\matQ}{\mathbb{Q}}

% Used for set
\newcommand{\setI}{\mathcal{I}}
\newcommand{\setJ}{\mathcal{J}}
\newcommand{\setK}{\mathcal{K}}
\newcommand{\setL}{\mathcal{L}}

% Users and Buckets
\newcommand{\setU}{\mathcal{U}}
\newcommand{\setC}{\mathcal{C}}

\newcommand{\setA}{\mathcal{A}}
\newcommand{\setW}{\mathcal{W}}

% Used for set
\newcommand{\setN}{\mathcal{N}}
\newcommand{\setS}{\mathcal{S}}


\newenvironment{disarray}%
 {\everymath{\displaystyle\everymath{}}\array}%
 {\endarray}

 \newcommand{\pdi}[1]{\frac{\partial #1}{\partial x_i}}
 \newcommand{\pdj}[1]{\frac{\partial #1}{\partial x_j}}

% Package for date and time presentation
\usepackage{datetime}
\newcommand{\datation}{\today, \xxivtime}

\def\full{0}        % set 1 for a full tech report version
                    % set 0 for submission version
\def\shownotes{1}   % set 1 for version with author notes
                    % set 0 for no notes
\def\anon{1}        % set 1 to anonymize
                    % set 0 for acks and author names

%%%%%%%  Author Notes %%%%%%%
%
\ifnum\shownotes=1
\newcommand{\authnote}[2]{{ $\ll$\textsf{\footnotesize #1 notes: #2}$\gg$}}
\else
\newcommand{\authnote}[2]{}
\fi
\newcommand{\Anote}[1]{{\authnote{Andrius}{#1}}}

%%%%%%%%%%%%%%%%%%%%%%%%%%%%%%%%%

\newcommand{\namedref}[2]{#1~\ref{#2}}
\newcommand{\tableref}[1]{\namedref{Table}{#1}}
\newcommand{\sectionref}[1]{\namedref{Section}{#1}}
\newcommand{\appendixref}[1]{\namedref{Appendix}{#1}}
\newcommand{\theoremref}[1]{\namedref{Theorem}{#1}}
\newcommand{\remarkref}[1]{\namedref{Remark}{#1}}
\newcommand{\definitionref}[1]{\namedref{Definition}{#1}}
\newcommand{\figureref}[1]{\namedref{Figure}{#1}}
\newcommand{\lemmaref}[1]{\namedref{Lemma}{#1}}
\newcommand{\claimref}[1]{\namedref{Claim}{#1}}
\newcommand{\propositionref}[1]{\namedref{Proposition}{#1}}
\newcommand{\constructionref}[1]{\namedref{Construction}{#1}}
\newcommand{\corollaryref}[1]{\namedref{Corollary}{#1}}
\newcommand{\equationref}[1]{\namedref{Equation}{#1}}
%
\newtheorem{theorem}{Theorem}[section]
\newtheorem{definition}[theorem]{Definition}
\newtheorem{lemma}[theorem]{Lemma}
\newtheorem{claim}[theorem]{Claim}
\newtheorem{obs}[theorem]{Observation}
%


\providecommand{\vs}{vs. }
\providecommand{\ie}{\emph{i.e.,} }
\providecommand{\eg}{\emph{e.g.,} }
\providecommand{\cf}{\emph{cf.,} }
\providecommand{\resp}{\emph{resp.,} }
\providecommand{\etal}{\emph{et al.}}   %Removed trailing space here; usually want non-breaking space with following reference
\providecommand{\etc}{\emph{etc.}}      % No trailing space here either
\providecommand{\mypara}[1]{\smallskip\noindent\emph{#1} }
\providecommand{\myparab}[1]{\smallskip\noindent\textbf{#1} }
\providecommand{\myparasc}[1]{\smallskip\noindent\textsc{#1} }
\providecommand{\para}{\smallskip\noindent}

\newtheorem{axiom}{{\bf  Axiom}}
\newtheorem{defin}{{\bf  Definition}}
\newtheorem{proposition}{Proposition}

\usepackage{enumitem}
\setlist{nolistsep}

%Model parameters

\usepackage[breaklinks=true]{hyperref}
\frenchspacing
\begin{document}

%don't want date printed
\date{}


%make title bold and 14 pt font (Latex default is non-bold, 16 pt)
\title{noTCP: extend, extend, repurpose}
\ifnum\anon=1
\author{[Paper: \#XXX]}% \hspace{0.2cm} \ampmtime ]}
\else
\numberofauthors{2}
\author{
\alignauthor Andrius Aucinas\\
\affaddr{University of Cambridge} 
\and
\alignauthor Jon Crowcroft\\
\affaddr{University of Cambridge}
}
\fi
%for single author (just remove % characters)

    
% end author
\maketitle
% Use the following at camera-ready time to suppress page numbers.
% Comment it out when you first submit the paper for review.
%\thispagestyle{empty}
\begin{abstract}
Too abstract.
\end{abstract} 

\section{Introduction}
\label{section:intro}

Extensibility of the Internet has been not like originally envisioned for a long time now, starting with IP layer and extending the \emph{narrow waist} to TCP or even HTTP at large. The problem primarily arises from the various middleboxes that perform packet content processing withing the network to improve performance\cite{Kopparty:2002ht,Chakravorty:2003dm}, increase security\cite{Handley:2001vp,Vutukuru:2008fc} and solve address shortage problems and is already well documented~\cite{UntoldMiddlebox2011,Qian:2012bj,Honda:2011ci,Guha:2005tb}.

The lack of extensibility has become acutely noticeable with the development of TCP extensions such as MPTCP and TCPCrypt, the former to make use of multi-homed endpoints, particularly useful in wireless networks and the latter to provide ubiquitous end-to-end traffic encryption. Even congestion control, the core feature of TCP, has required multiple extensions since protocol's inception and the large number of proposed congestion control schemes~\cite{5462976} suggests that more may be needed. These and other extensions are primarily designed around the limitations of existing middleboxes, but they demonstrate two problems: 1) any extension design is extremely constrained by prevalent middlebox behaviors\cite{Honda:2011ci} and 2) we have already run out of TCP option space, the built-in mechanism for extending TCP.

Problems with the size of option space arise in MPTCP when its options are combined with those of regular TCP and therefore needs to change the semantics of a duplicate ACK~\cite{Handley:vj} and there are efforts to define an alternative control stream to work around the limitation of option space~\cite{Bonaventure:wx}. TCPCrypt goes even further and repurposes the Data portion of TCP segments to exchange cryptographic keys~\cite{Mazieres:uz}.

We propose \emph{noTCP}, an approach complementary to option negotiation which repurposes TCP header fields beyond what is specified but at the same time to be compatible with most of existing, deployed network infrastructure and safely fall-back to vanilla TCP if such modifications fail. At the core of the idea is modification of TCP header fields in a way that is not interfered with by middleboxes or so that such interference can be safely recovered or ignored.

With the focus on cellular and WiFi networks due to their renown middlebox peculiarities, we make the following contributions in this work:

\begin{enumerate}
    \item Provide an analysis of current networks behavior in the presence of unspecified TCP protocol behavior.
    \item Propose a robust method to negotiate endpoint-specific behavior across middleboxes.
    \item Demonstrate that the method is effective across a number of cellular and WiFi networks.
\end{enumerate}

\section{Candidate protocol changes}

Without using the options field and only changing the meaning of header fields in a backwards-compatible manner we only have a few options:
\begin{itemize}
    \item Setting header fields to specific values without setting the corresponding flags so that they do not have a meaning to legacy implementations.
    \item Assigning one or more of the reserved field bits.
    \item Defining semantics for specific values of currently valid header fields.
\end{itemize}

In this section we overview these different options, discuss their pros and cons and show how real networks interact with them.

\subsubsection*{Unset Flags}

The only header fields that may or may not have a meaning depending on the corresponding flag are \emph{urgent pointer} and \emph{acknowledgment number},with the latter only not being set at the initial SYN packet.

Urgent pointer is both a good and a bad candidate at the same time: it is already recommended against the use of the urgent mechanism~\cite{Gont:2011vi}, therefore it could be claimed that it is taking up precious space in the TCP header. On the other hand, it is common practice by Network Intrusion Detection Systems (NIDS) to reset the field as it makes it difficult to track the application-layer data~\cite{seolma}. We present our observations of how often the field is filtered in Section~\ref{sec:network}, however although we witnessed the field to be passed through on many of our studied networks and hence it is a usable option, it is not without drawbacks.

Acknowledgment field, on the other hand, is double the length and we saw it successfully pass through the networks in more cases. Nevertheless, unless specific value is signaled out of bound between the two endpoints to use the field with TCP Simultaneous Open, only the initiating end can use it and the responding side needs to acknowledge the understanding of it in some other method. 

\subsubsection*{Reserved bits}

The main difficulty with TCP Reserved bits is that there are only 3 bits unallocated if we consider the three ECN bits added over 10 years ago~\cite{Floyd:up,Ely:uc} but still not universally implemented~\cite{}. Any further reserved bit allocation would therefore be extremely difficult.

Nevertheless, there are proposals that make use of them. One of the proposals for reducing Web latency~\cite{Flach:2013uy} suggests sending multiple copies of packets, but to avoid triggering duplicate acknowledgments add a reserved bit to the header. We confirmed the authors observation that middleboxes do not seem to discard packets with non-compliant TCP flags, however we also saw a number of middleboxes to clear such flags, hence interfering with the duplicate acknowledgment of the proposed scheme. Due to the problem with reserved bit allocation we would recommend using mechanisms we discuss later.

\subsubsection*{Value-specific semantics}

Our remaining class of options, and by far the biggest, is defining protocol semantics based on specific values of certain header fields. It would necessitate generating crafting packets with specific values for initial sequence numbers, window size or a specific checksum value, and we posit that it would be an acceptable choice if such solution is deployable in today's network and can be implemented efficiently.

Initial sequence numbers, must be hard to predict as they provide a measure (however weak) of security and until a stronger one (such as TCPCrypt) is universally adopted adding any information to the field in a way that is easily decoded by the receiver weakens security.

Window size field is potentially a better candidate, especially if used only during connection setup phase, since data is not typically transmitted during the handshake. We could further claim that even during data exchange a few of the least-significant bits could be sacrificed for carrying other information, however that would prevent flow control from operating properly. Higher-order bits could instead be given up when window scaling option is used to compensate for the decrease of signaled window size, but for every such bit the maximum window would be halved, having negative effects for more capable clients connected though high capacity links. Finally, we observed many middleboxes to modify the field and there is a risk of the middleboxes using the initially negotiated one for assigning resources for the duration of the connection~\cite{}.

There are multiple benefits of using the checksum field. Firstly, it is (only) used for error detection as opposed to other fields that govern protocol behavior and is either not always necessary due to the layered design of the OSI stack (when error detection is done at the layer below or above), or can be replaced with extensions' separate (MPTCP DSS checksum) or even stronger (TCPCrypt MAC) mechanisms. Secondly, checksum operations are efficient as they only involve one's complement addition. And finally, it is possible to generate checksums in a way that the intended value can recovered by the receiving end after traversing common types of middleboxes that recompute the checksum as we discuss in the following sections.


\section{Methodology}

To test what works on current networks we have developed a simplified implementation of TCP that exchanges packets in the standard order (three-way handshake, data exchange and connection shutdown) but with various header fields modified to our purposes.

The test has been implemented as an Android application and distributed through the \emph{Google Play Store} to collect data from volunteers. The overall structure of the tool is relatively straightforward:
\begin{itemize}
    \item UI which reports aggregated results to the end-users, sends results to the reporting server and manages the overall flow of the tests.
    \item Native implementation of the simplified TCP stack using Raw Unix sockets (\emph{SOCK\_RAW}) to allow us to generate packets at the network layer. This also meant having to drop RST packets generated by the kernel network stack during the test in response to incoming packets associated with connections it is not aware of - packets are delivered to both the raw socket and the kernel network stack.
    \item Similar server implementation that generates specific answers to incoming packets based on header and payload values.
\end{itemize}

Potential limitations of our methodology include that we do not consider packet resegmentation since the goal is to negotiate any endpoint-specific in packet headers so that correct behavior does not depend on how packets are split or coalesced together. To make sure that it does not happen all our packets contain minimal amount of payload (up to 16 bytes during data exchange) and data exchange only consists of a single message sent each way, in addition to any extra acknowlegements generated by middleboxes. Due to the same reason we also do not take into account congestion control.

Instead of trying to infer which source port number would be assigned to a new connection by the kernel network stack without using it, we randomised the selection within the unprivileged port range. It was a simple optimisation that works on any device and minimises the probability of multiple tests failing due to another application using a particular port. We also repeated the tests on different destination ports, which we discuss in Sec.~\ref{sec:portspec}.

\section{Network cahracteristics}
\label{sec:network}

\begin{table*}[t]
{\small
\begin{center}
\begin{tabular}{| l | >{\centering\arraybackslash}m{0.8cm} | >{\centering\arraybackslash}m{0.8cm} | >{\centering\arraybackslash}m{0.9cm} | >{\centering\arraybackslash}m{1.6cm} | >{\centering\arraybackslash}m{1.6cm} | >{\centering\arraybackslash}m{1.6cm} | >{\centering\arraybackslash}m{1.5cm} | >{\centering\arraybackslash}m{1.5cm} | >{\centering\arraybackslash}m{1.5cm} | >{\centering\arraybackslash}m{1.cm} | }
\hline
    \textbf{Net ID} & \textbf{Country} & \textbf{No filter} & \textbf{Global IP}   & \textbf{Port-specific} & \textbf{Validate checksum} & \textbf{Drop S/A data} & \textbf{Normalize ACK}  & \textbf{Normalize URG} & \textbf{Normalize Reserved} & \textbf{Remap Seq.} \\ \hline \hline
    % Univ Helsinki WiFi
    WiFi edu 1      & FI               &                    &                      &                        & \checkmark \checkmark      &                        &                         &                        &                             &                     \\ \hline
    % Eduroam
    WiFi edu 2      & UK               &                    & \checkmark           &                        &                            & \checkmark             &                         & \checkmark \checkmark  &                             & \checkmark          \\ \hline
    % wgb
    WiFi pub 1      & UK               & \checkmark         &                      &                        &                            &                        &                         &                        &                             &                     \\ \hline
    % MKSW germany (carlos)
    WiFi pub 2      & DE               &                    &                      &                        & \checkmark \checkmark      & \checkmark             &                         &                        &                             &                     \\ \hline
    % Virgin home
    WiFi res 1      & UK               &                    &                      &                        & \checkmark \checkmark      &                        &                         &                        &                             &                     \\ \hline
    \hline
    % Finland internet.saunalahti
    Cellular 1      & FI               & \checkmark         &                      &                        &                            &                        &                         &                        &                             &                     \\ \hline
    % GiffGaff
    Cellular 2      & UK               &                    &                      &  \checkmark (443, 993) & -                          & -                      & -                       & -                      & -                           & -                   \\ \hline
    % E-Plus germany
    Cellular 3      & DE               &                    &                      &                        &                            & \checkmark             &                         & \checkmark \checkmark  &                             & \checkmark          \\ \hline
\end{tabular}
\end{center}
}
\caption{Network behaviour observed through tests generating custom TCP packets. A single checkmark means that particular behaviour was observed or in case of normalization the field was reset or replaced with a valid value. Double checkmark means that specific type of packet was discarded. A dash means that we observed different cases on the same network.}
\label{tab:networks}
\end{table*}

We show our collected results in Table~\ref{tab:networks}. Since we were focusing on mobile applications, we collected data across a number of public, residential and university WiFi networks and cellular networks in XX different countries: UK, USA, Finland (pending: Turkey, India, Lithuania, Spain...). We picked the most important dimensions that show what modifications to the protocol have the highest chance of being successull across the largest number of networks. In particular, the results include which networks provided clients a global IP address (very few), which ones we did not see any filtering for (a relatively large, but insufficient number to take any deployment for granted), and which ones discarded packets with invalid checksums, normalised particular fields - either dropped packets or reset to different values - and what type of mapping did NATs perform - only port number remapping or also changed sequence numbers.

Checksum behaviour was particularly interesting. In traditional NATs~\cite{Egevang:tu} checksum recomputation is very simple: subtract the old header fields from the checksum and add the new ones to minimize the resource use. In our sample X networks only do that, as indicated by the fact that they allow packets with incorrect checksums through towards the endpoint (otherwise the packet would either be dropped or a valid checksum would be recomputed). It is not always the case and another alternative is to add payload to a packet to make checksum a specific value. It can only be done when the receiving end is already expecting it, since it should not pass such payload up to the layer above, therefore can only be used from SYNACK onwards. A bigger issue here is that middleboxes do not always respect the standard and simply drop such packets - a surprising 50\% of our tested networks - even though the standard clearly allows handshake packets to contain payload~\cite{Postel:3EDyoxP_,Chu:2011tn}:

\begin{quotation}
    Although these examples do not show connection synchronization using data-carrying segments, this is perfectly legitimate
\end{quotation}

We also tested middlebox itneraction with reserved header bits. We separated our tests to handshake and data exchange parts as well as set each bit individually, however we did not observe different behavior for these: all reserved bits were either allowed through or cleared at both phases. Importantly, we did not observe cases when such packets are dropped.


\subsection{Port-specific middlebox behavior}
\label{sec:portspec}

One column of our table of observations is especially interesting -- port-specific behavior. We run our tests on a number of different ports:
\begin{itemize}
    \item 80 - HTTP port allowed through most firewalls
    \item 443 - HTTPS port
    \item 993 - Secure IMAP port
    \item 5228 - Google cloud messaging port
    \item 6969 - random port number
    \item 8000 - common HTTP proxy port number
\end{itemize}

So far we have only observed port-specific behavior on one network, \emph{Cellular 2} but the particular network appears to be configured to allow traffic to commonly used SSL ports (443, 993) pass through unmodified - most of our attempted modifications are allowed through with the exception of packets that do not have a valid checksum being dropped. There may be other ports configured to allow unmodified traffic through, but we only focused on commonly used ports.

All other traffic, however, is intercepted at the middlebox which we have separately confirmed to be a Bytemobile proxy~\footnote{Citrix Bytemobile - is it appropriate to reference this case accurately?} and it terminates clients' connections, establishing new ones to the destination, as well as adding options not present in original packets, changing window size, assigning new source port numbers sequentially and even postponing handshaking until the handshake with the client is complete. Such behavior makes deployment of any protocol modifications or extensions very difficult (e.g. we verified that TCPCrypt does not work on this network) and although we did not see many cases of it, the product's website claims that they have over 160 clients, which suggests that it may be common.

In addition to noting that our traffic was not encrypted but only sent to a specific destination port, it is also important to say that an increasing proportion of traffic is served over SSL. A commerciall deployed opt-in HTTP accelerator reports the proportion of HTTPS requests to be close to that of HTTP, although varying with time~\footnote{http://db.awazza.com/users/global/}. 

\subsection{Proxies and NATs}

Based on results from Netalyzr?

\section{Protocol negotiation}

What is common to all candidate protocol changes described above is that they are very limited in size and therefore can only be used to exchange an opcode rather and not a substantial amount of information. Such information must instead be exchanged through other means - either implicit to the opcode, or explicit by redefining the meaning of part of the data payload or previously exchanged information. We therefore focus on combining the above observations into a robust exchange of the opcode between two endpoints in current networks.

\section{Discussion}

The main points of discussion on \emph{noTCP} is the same as with all TCP extensions: is it necessary, is it deployable and is it forwards-compatible? Furthermore, any special protocol negotiation still needs to be standardized to ensure compatibility at the Internet scale...

Nevertheless, it is important 



\section{Related Work}
\label{sec:related}

TCP implements an \emph{urgent mechanism} at its core that allows the sending user to stimulate the receiving user to accept some \emph{urgent data}. The mechanism is often quoted as providing ``out-of-bound'' data delivery even though it is specified that it is not a mechanism for sending such data. Furthermore, there are ambiguities regarding the semantics of the urgent pointer, Network Intrusion Detection Systems (NIDS) tend to clear the URG flag and pointer and in general it is recommended against the use of the mechanism~\cite{Gont:2011vi}.

In the context of MPTCP one proposal to add a generic control stream is to map such stream into a separate sequence number space and exchange control data over established subflows, only modifying MPTCP DSS option to use one of the reserved bits to differentiate the control stream from the data stream. Arguably the specified options are complex as well as extensible to enable it are precisely because there are very few ways in which TCP itself can be extended. the underlying assumption, however, is that theyt MPTCP is already deployed as well as that existing deployments will be forwards-compatible with the specification.

Conceptually close to MPTCP's subflows is SCTP's \emph{data chunks}, where a packet can contain multiple TLV (Type-Length-Value) encoded chunks to separate user data and control information. Despite its challenged deployment in the Internet, SCTP over DTLS is being used as the basis of WebRTC data protocol~\cite{Tuexen:wv} and we suggest that such chunking could be used on top of \emph{noTCP} once such behavior has been negotiated as described above.


\section{Conclusion}
TBD
%\input{conclusion}

\ifnum\anon=0
\subsection*{Acknowledgments}

\fi


\clearpage
%Only show things we really cite :)
% \nocite{*}
{\footnotesize 
\bibliographystyle{abbrv}
%\footnotesize
\bibliography{references}
}

%\appendix 
%\input{data}

\end{document}







